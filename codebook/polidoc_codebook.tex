\documentclass[a4paper, 12pt]{article}  
\usepackage[margin=2.54cm]{geometry}
\renewcommand{\baselinestretch}{1.5}
\usepackage{amsmath}
\usepackage{url} 
\usepackage{graphicx}
\usepackage[ansinew]{inputenc}
\usepackage{pdflscape}
%\usepackage{lscape}
\usepackage{longtable}

\begin{document}

\title{Polidoc.net \\
CODEBOOK \\
\large
\medskip
National and Regional Manifestos and other Political Documents Collected for the Research Projects
\\
"Representation in Europe: Congruence between Preferences of Elites and Voters" \\ (REPCONG)
\\
\textit{and} 
\\
"The Impact of EU Cohesion Policy on European Identification" \\ (COHESIFY)
} 

\date{
% June 27, 2018 
\today
}

\author{
\small
Prepared by the Polidoc.net Team headed by \\ \small
 Thomas Bräuninger (University of Mannheim), \\ \small
 Marc Debus (University of Mannheim), \\ \small
 Kenneth Benoit (London School of Economics and Political Science) \\ \small
 and Julian Bernauer (Mannheim Centre for European Social Research) \\ \small
doi (of November 21, 2017 version): \url{http://dx.doi.org/10.5281/zenodo.1067701}
}

\maketitle

\clearpage

\section{polidoc.net- The Political Documents Archive}
The Political Documents Archive contains election manifestos, coalition agreements, government declarations and various other documents of political actors from developed democracies. Currently, the archive builds on a stock of more than 3000 political documents (June 27, 2018: 1230 national and 2184 subnational manifestos) from currently 26 European countries. The aim of the repository is to provide political texts in order to facilitate scholarly research in different areas of comparative politics such as party competition, coalition politics, legislative decision-making or electoral behavior. National electoral manifestos have been collected in the course of the REPCONG project, and the archive includes party manifestos for regional elections in several European democracies. Because the process of European integration resulted in a strengthening of regions in EU member states and in countries that want to join the European Union, the relevance of the regional level for political decision-making has increased during the last decades. Therefore, also the policy profiles of regional parties are required to get a full picture of democratic responsiveness in European states across all levels of the political system. The collection of regional manifestos and national manifestos since 2013 (currently: 2427 texts) was supported by the COHESIFY project (www.cohesify.eu), funded under the Horizon 2020 Framework Programme for Research and Innovation. The aim of COHESIFY is to study whether the European Structural and Investment Funds affect people’s support for and identification with the European project.

The archive is freely accessible (after a simple registration) and meant to foster rigorous research in these areas by enabling scholars to produce valid and reliable findings from empirical studies of textual data rather than unnecessarily struggling to obtain and process texts. While polidoc.net is an open access archive it also aims to establish a network for exchanging and accessing political text data once it has been checked for errors and assured for quality. The archive therefore encourages researchers to share their documents with others. We invite everybody to participate in this project so that the political science community can easily make use of already collected documents.

\clearpage

\section{Countries, elections and parties covered}
The REPCONG project has collected national electoral manifestos of parties in 20 countries until 2012: Austria, Belgium, Czech Republic, Denmark, Estonia, Finland, France, Germany, Ireland, Italy, Luxembourg, Malta, Netherlands, Norway, Poland, Portugal, Spain, Sweden, Switzerland and the United Kingdom. 
The COHESIFY project has focused on Germany, Greece, Hungary, Ireland, Poland, Romania, Slovenia, Italy, Netherlands, UK and Spain. Polidoc.net currently houses regional documents collected in the course of the project from Austria, Belgium, Netherlands, Sweden, Switzerland, Czech Republic, Germany, Spain, Italy (Lombardy) and the United Kingdom. 
The time period covered mainly encompasses 1980 onwards for national manifestos and 2000 onwards for regional documents, partially reaching farther back. See below for details. 

The selection rules for national documents from western European countries are: 
\begin{enumerate}
\item 	Include all parties with at least 1\% of the (valid) votes in the election for which the manifesto was written
\item 	If the sum of votes covered is less than 95\%, include the next largest parties until 95\% are covered
\item 	For all parties that sometimes are in the selection and sometimes not, include one additional observation before and after their appearance(s)
\end{enumerate}
For the post-communist European countries, only steps 1 and 2 from above apply.


\section{Sources}
The main sources include: 
\begin{itemize}
\item Zentralarchiv für Empirische Sozialforschung (ZA)
\item The Comparative Manifesto Project (CMP) group
\item Internet resources of parties 
\item German subnational manifestos before 1990: Prof. Franz Urban Pappi (University of Mannheim)
\item Other personal contacts
\end{itemize}

%Details...

\section{Archive and file Structure }
The structure of the files is as follows: Each file is labeled with a 5-digit code. The codes are based on the rules used by the CMP group as described in Budge et al. (2001: 193).\footnote{Budge, Ian, Hans-Dieter Klingemann, Andrea Volkens, Judith Bara and Eric Tanenbaum. 2001. \textit{Mapping Policy Preferences: Estimates for Parties, Electors and Governments} 1945-1998. Oxford: Oxford University Press.} The first two numbers indicate the country, the next two numbers indicate the party family, while the last one is for the number of the party if several parties of the same party family are coded. Codes for parties previously not covered by the CMP group have been extended to previously uncoded parties following the coding rules of the CMP. 


\section{Formatting Rules for Manifestos}

\begin{enumerate}
\item Starting from a PDF, removal of page numbers, headers, footnotes, donation and membership calls as well as indexes and registers, keeping titles and forewords. 
\item Removal or correction of unusual characters and formatting resulting from flawed scans or conversions (comparison with PDF of the manifesto).
\item Check for and add missing sentences and paragraphs (if possible from original source, preferably PDF).
\item Reassembling of hyphenations.
\item Substituting bullet points, enumerations and other structuring characters including Roman and Arabic numerals with a dash (-).
\item (Uniform) spell check (Word).
\item Saving of corrected manifestos as non-formatted text files (.txt) with UTF-8 coding. The format grants cross-platform use of the texts.
\end{enumerate}

\newpage

\section{Citation}
The archive is maintained by Thomas Bräuninger (University of Mannheim), Marc Debus (University of Mannheim), Kenneth Benoit (London School of Economics and Political Science) and Julian Bernauer (Mannheim Centre for European Social Research). 
The codebook has its own doi: \url{http://dx.doi.org/10.5281/zenodo.1067701} and is hosted on Zenodo (see \url{https://zenodo.org/record/1067701}). 
If data are used in publications etc. please refer to the Political Documents Archive as the data source by citing:

\begin{itemize}
\item Kenneth Benoit, Thomas Bräuninger and Marc Debus. 2009. "Challenges for estimating policy preferences: Announcing an open access archive of political documents." \textit{German Politics} 18(3): 440-453.
%, http://www.tandfonline.com/doi/abs/10.1080/09644000903055856.
\item Please refer additionally to the following article when using data of parties from the regional level: Martin Gross and Marc Debus. 2017. "Does EU regional policy increase parties' support for European integration?" \textit{West European Politics} (forthcoming).
%, http://www.tandfonline.com/doi/full/10.1080/01402382.2017.1395249.
\item And for German subnational manifestos before 1990: Pappi, Franz Urban and Nicole Seher. 2014. "Die Politikpositionen der deutschen Landtagsparteien und ihr Einfluss auf die Koalitionsbildung". In: Eric Linhart, Bernhard Kittel, André Bächtiger (Eds), \textit{Räumliche Modelle der Politik}. Wiesbaden: Springer VS, 171-205; as well as: Pappi, Franz Urban and Nicole Seher. 2009. "Party Election Programmes, Signalling Policies and Salience of Specific Policy Domains: The German Parties from 1990 to 2005". \textit{German Politics}, 18(3): 403-425.
\end{itemize}

\newpage

\section{Feedback}
Please contact us using the form on \url{http://www.polidoc.net}.

\section{Coverage}
The remainder of the documentation provides 1) national and regional lists of party coverage with unique codes for the single parties (using CMP-style codes as a point of departure as described above) and full (English) party names as well as 2) national and regional graphs providing information on the elections covered per party. To locate and download documents, please register and use the query interface on polidoc.net. 


\clearpage

\begin{landscape}

%\subsection{National documents}

%add "`longtable"' in file 
% latex table generated in R 3.4.1 by xtable 1.8-2 package
% Wed Jun 27 10:12:34 2018
\begin{footnotesize}
\begin{longtable}{lrll}
\caption{Parties issuing national manifestos covered} \\
Country & CMP code & Acronym & Party \\ 
  \hline
Sweden & 11110 & MP & Green Ecology Party \\ 
  Sweden & 11220 & VP & Left Party \\ 
  Sweden & 11320 & SAP & Social Democratic Labor Party \\ 
  Sweden & 11420 & FP & Liberal People�s Party \\ 
  Sweden & 11520 & KdS & Christian Democratic Community Party \\ 
  Sweden & 11620 & MSP & Moderat Coalition Party \\ 
  Sweden & 11701 & NULL & NULL \\ 
  Sweden & 11710 & SD & Sweden Democrats \\ 
  Sweden & 11810 & CP & Centre Party \\ 
  Sweden & 11951 & NyD & New Democracy \\ 
  Norway & 12220 & NKP & Norwegian Communist Party \\ 
  Norway & 12221 & SV & Socialist Left Party \\ 
  Norway & 12320 & DANN & Labour \\ 
  Norway & 12410 & DLF & Liberal Peoples Party \\ 
  Norway & 12420 & V & Liberal Party \\ 
  Norway & 12421 & NULL & NULL \\ 
  Norway & 12520 & KrF & Christian People�s Party \\ 
  Norway & 12620 & H & Conservative Party \\ 
  Norway & 12810 & SP & Centre Party \\ 
  Norway & 12951 & FrP & Progress Party \\ 
  Denmark & 13001 & NULL & New Alliance \\ 
  Denmark & 13210 & Y & Left Socialists \\ 
  Denmark & 13221 & P & Common Course \\ 
  Denmark & 13229 & Enh & Red-Green Alliance \\ 
  Denmark & 13230 & SF & Socialist People's Party \\ 
  Denmark & 13320 & SD & Social Democratic Party \\ 
  Denmark & 13330 & CD & Centre Democrats \\ 
  Denmark & 13401 & Y & New Alliance \\ 
  Denmark & 13410 & RV & Danish Social Liberal Party \\ 
  Denmark & 13420 & V & Liberals \\ 
  Denmark & 13520 & K & Christian Democrats \\ 
  Denmark & 13620 & KF & Conservative People's Party \\ 
  Denmark & 13701 & NULL & NULL \\ 
  Denmark & 13720 & DF & Danish Peoples Party \\ 
  Denmark & 13951 & FrP & Progress Party \\ 
  Denmark & 13952 & E & Justice Party \\ 
  Finland & 14110 & VL & Green Union \\ 
  Finland & 14221 & SKDL & Finnish People�s Democratic Union \\ 
  Finland & 14222 & DEVA & Democratic Alternative \\ 
  Finland & 14223 & Vas & Left Alliance \\ 
  Finland & 14320 & SSDP & Finnish Social Democrats \\ 
  Finland & 14430 & NSP & Progressive Finnish Party \\ 
  Finland & 14520 & SKL & Finnish Christian Union \\ 
  Finland & 14620 & KK & National Coalition \\ 
  Finland & 14810 & SK & Finnish Centre \\ 
  Finland & 14820 & PS & True Fins \\ 
  Finland & 14901 & RKP/SFP & Swedish People�s Party \\ 
  Belgium & 21111 & Ecolo & Francophone Ecologists \\ 
  Belgium & 21112 & Groen & Green! \\ 
  Belgium & 21221 & SPA & Spirit - Socialist Party Different \\ 
  Belgium & 21321 & SP & Flemish Socialist Party \\ 
  Belgium & 21322 & PS & Francophone Socialist Party \\ 
  Belgium & 21402 & Vivant &  \\ 
  Belgium & 21403 & LDD & List Dedecker \\ 
  Belgium & 21421 & VLD & Flemish Liberals and Democrats \\ 
  Belgium & 21422 & PRL & Francophone Liberals \\ 
  Belgium & 21423 & FDF & Democratic Front of the Francophones \\ 
  Belgium & 21425 & PRL-FDF-MCC & Liberal Reformation Party - ... \\ 
  Belgium & 21426 & MR & Reformist Movement \\ 
  Belgium & 21521 & CVP & Christian Peoples Party \\ 
  Belgium & 21522 & PSC & Christian Social Party \\ 
  Belgium & 21523 & CD\&V & Christian Democratic and Flemish \\ 
  Belgium & 21524 & CDH & Humanist Democratic Centre \\ 
  Belgium & 21911 & RW & Walloon Rally \\ 
  Belgium & 21912 & FDF & Francophone Democratic Front \\ 
  Belgium & 21913 & VU & Peoples Union \\ 
  Belgium & 21914 & VB & Flemish Bloc \\ 
  Belgium & 21915 & VU-ID21 & People's Union - Ideas for the 21th Century \\ 
  Belgium & 21916 & N-VA & New-Flemish Alliance \\ 
  Belgium & 21917 & FNB & New Belgian Front \\ 
  Netherlands & 22110 & GL & Green Left \\ 
  Netherlands & 22201 & CPN & Communist Party of the Netherlands \\ 
  Netherlands & 22202 & PSP & Pacifist Socialist Party \\ 
  Netherlands & 22220 & SP & Socialist Party \\ 
  Netherlands & 22310 & PPR & Political Party Radicals \\ 
  Netherlands & 22320 & PvdA & Labour Party \\ 
  Netherlands & 22330 & D66 & Democrats '66 \\ 
  Netherlands & 22420 & VVD & People's Party for Freedom and Democracy \\ 
  Netherlands & 22430 & LN & Livable Netherlands \\ 
  Netherlands & 22501 & GPV & Reformed Political Union \\ 
  Netherlands & 22502 & SGP & Political Reformed Party \\ 
  Netherlands & 22503 & RPF & Reformational Political Federation \\ 
  Netherlands & 22521 & CDA & Christian Democratic Appeal \\ 
  Netherlands & 22525 & NULL & NULL \\ 
  Netherlands & 22526 & CU & ChristianUnion \\ 
  Netherlands & 22601 & PVV & Party for Freedom \\ 
  Netherlands & 22701 & CD & Center Democrats \\ 
  Netherlands & 22720 & LPF & List Pim Fortuyn \\ 
  Netherlands & 22906 & 50+ & 50PLUS \\ 
  Netherlands & 22951 & AOV & General Union for Elderly People \\ 
  Netherlands & 22952 & PvdD & Party for the Animals \\ 
  Netherlands & 22953 & U55 & Union 55+ \\ 
  Luxembourg & 23111 & GLEI & Green Left Ecological Initiative \\ 
  Luxembourg & 23112 & GAP & Green Alternative \\ 
  Luxembourg & 23113 & GLEI-GAP & Green Left Ecological Alternative - Green Alt. \\ 
  Luxembourg & 23114 & NULL & The Greens \\ 
  Luxembourg & 23202 &  & The Left \\ 
  Luxembourg & 23220 & NULL & NULL \\ 
  Luxembourg & 23320 & POSL/LSAP & Socialist Workers� Party \\ 
  Luxembourg & 23420 & PD/DP & Democratic Party \\ 
  Luxembourg & 23520 & PCS/CSV & Christian Social People�s Party \\ 
  Luxembourg & 23601 & ADR & Alternative Democratic Reform Party \\ 
  France & 31110 & Verts & The Greens \\ 
  France & 31111 & Eco & Ecology Generation \\ 
  France & 31220 &  & French Communist Party \\ 
  France & 31320 & PS & Socialist Party \\ 
  France & 31621 & Gaul & Gaullists \\ 
  France & 31622 & NULL & NULL \\ 
  France & 31624 & UDF & Union for French Democracy \\ 
  France & 31625 & RPR & Rally for the Republic \\ 
  France & 31626 & UMP & Union for a Popular Movement \\ 
  France & 31702 & RP & Republican Pole \\ 
  France & 31703 & MNR & National Republican Movement \\ 
  France & 31720 & FN & National Front \\ 
  France & 31981 & RPR / UDF & Common list RPR / UDF \\ 
  Italy & 32110 & FdV & Green Federation \\ 
  Italy & 32211 & DP & Proletarian Democracy \\ 
  Italy & 32212 & PRC & Newly Founded Communists \\ 
  Italy & 32213 & NULL & NULL \\ 
  Italy & 32220 & DS & Democrats of the Left \\ 
  Italy & 32310 &  &  \\ 
  Italy & 32320 & PSI & Socialist Party \\ 
  Italy & 32330 & PSDI & Italian Democratic Socialist Party \\ 
  Italy & 32340 & PD & Democratic Party \\ 
  Italy & 32404 & IDV & Italy of Values \\ 
  Italy & 32410 & PRI & Republican Party \\ 
  Italy & 32420 & PLI & Liberal Party \\ 
  Italy & 32430 & SC & Civic Choice \\ 
  Italy & 32520 & PPI & Italian Peoples Party \\ 
  Italy & 32521 & CCD & Christian Democratic Centre \\ 
  Italy & 32522 & NULL & NULL \\ 
  Italy & 32528 & PI & Pact for Italy \\ 
  Italy & 32530 & UDC & Centre Union \\ 
  Italy & 32610 & FI & Forward Italy \\ 
  Italy & 32629 &  & House of freedom \\ 
  Italy & 32710 & AN & National Alliance \\ 
  Italy & 32720 & LN & Northern League \\ 
  Italy & 32902 &  &  \\ 
  Italy & 32951 & LR & The Network / Movement for Democracy \\ 
  Italy & 32993 & PDL & The People of Freedom \\ 
  Italy & 32995 &  &  \\ 
  Italy & 32996 & UNION & The Union \\ 
  Italy & 32997 & M5S & Five Star Movement \\ 
  Spain & 33220 & IU & United Left \\ 
  Spain & 33320 & PSOE & Spanish Socialist Workers� Party \\ 
  Spain & 33321 & NULL & NULL \\ 
  Spain & 33430 & UCD & Union of the Democratic Centre \\ 
  Spain & 33512 & CDS & Centre Democrats \\ 
  Spain & 33610 & PP & Popular Party \\ 
  Spain & 33611 & CiU & Convergence and Unity \\ 
  Spain & 33902 & PNV/EAJ & Basque Nationalist Party \\ 
  Spain & 33903 & EA & Basque Solidarity \\ 
  Spain & 33905 & ERC & Catalan Republican Left \\ 
  Spain & 33907 & CC & Canarina Coaliton \\ 
  Spain & 33908 & BNG & Galician Nationalist Bloc \\ 
  Spain & 33951 & NULL & NULL \\ 
  Spain & 33952 & NULL & NULL \\ 
  Spain & 33992 & UPyD & Union, Progress and Democracy \\ 
  Portugal & 35110 & PEV & Greens \\ 
  Portugal & 35210 & UDP & Popular Democratic Union \\ 
  Portugal & 35211 & BE & Left Bloc \\ 
  Portugal & 35220 & PCP & Portuguese Communist Party \\ 
  Portugal & 35310 & MDP & Democratic Movement \\ 
  Portugal & 35311 & PSP & Portuguese Socialist Party \\ 
  Portugal & 35312 & PRD & Democratic Renewal Party \\ 
  Portugal & 35313 & PSD & Social Democratic Party \\ 
  Portugal & 35314 & PPI & Popular Party \\ 
  Portugal & 35315 & NULL & NULL \\ 
  Portugal & 35520 & PP & Popular Party \\ 
  Portugal & 35710 & NULL & NULL \\ 
  Portugal & 35951 & PSN & National Solidarity Party \\ 
  Portugal & 35991 & FRS &  \\ 
  Portugal & 35992 & APU & United People Alliance \\ 
  Portugal & 35993 & CDU & Unified Democratic Coalition \\ 
  Portugal & 35995 & CDS-PP & Democratic and Social Center \\ 
  Germany & 41001 & NULL & Coalition Agreement \\ 
  Germany & 41111 & GR�NE & The Greens \\ 
  Germany & 41113 & GR�NE & Alliance�90/Greens \\ 
  Germany & 41220 & MLPD & Marxist�Leninist Party of Germany \\ 
  Germany & 41221 & PDS & Party of Democratic Socialism \\ 
  Germany & 41222 & L-PDS & The Left.Party of Democratic Socialism \\ 
  Germany & 41223 & LINKE & The Left \\ 
  Germany & 41230 & DKP & German Communist Party \\ 
  Germany & 41301 & WASG & Labour and Social Justice � The Electoral Alternative \\ 
  Germany & 41320 & SPD & Social Deomocratic Party of Germany \\ 
  Germany & 41420 & FDP & Free Democratic Party \\ 
  Germany & 41521 & CDU & Christian Democratic Union \\ 
  Germany & 41523 & CSU & Christian Social Union of Bavaria \\ 
  Germany & 41701 & REP & The Republicans \\ 
  Germany & 41702 & NPD & National Democratic Party of Germany \\ 
  Germany & 41703 & DVU & The Germans People Union \\ 
  Germany & 41950 & NULL & Pirate Party Germany \\ 
  Germany & 41953 & AfD & Alternative for Germany \\ 
  Austria & 42001 & �VP + Die Gr�nen  &  \\ 
  Austria & 42110 & GA & Green Alternative \\ 
  Austria & 42220 & KP� & Communist Party \\ 
  Austria & 42320 & SP� & Austrian Social Democratic Party \\ 
  Austria & 42420 & FP� & Freedom Movement \\ 
  Austria & 42421 & LF & Liberal Forum \\ 
  Austria & 42422 & BZ� & Alliance for the Future of Austria \\ 
  Austria & 42450 & neos & NULL \\ 
  Austria & 42520 & �VP & Austrian People�s Party \\ 
  Austria & 42710 & NULL & NULL \\ 
  Austria & 42951 &  & Hans-Peter Martin's List � For genuine control in Brussels \\ 
  Austria & 42952 & FRITZ &  \\ 
  Switzerland & 43110 & GPS & Green Party of Switzerland \\ 
  Switzerland & 43220 & PDA & Swiss Labour Party \\ 
  Switzerland & 43320 & SPS/PSS & Social Democratic Party \\ 
  Switzerland & 43321 & LdU/ADI & Ring of Independents \\ 
  Switzerland & 43401 & NULL & NULL \\ 
  Switzerland & 43420 & FDP/PRD & Radical Democratic Party \\ 
  Switzerland & 43520 & CVP/PDC & Christian Democratic People�s Party \\ 
  Switzerland & 43530 & EVP/PEP & Protestant Peoples Party \\ 
  Switzerland & 43531 & LPS & Liberal Party of Switzerland \\ 
  Switzerland & 43710 & SD & Swiss Democrats \\ 
  Switzerland & 43711 & EDU & Federal Democracy Union \\ 
  Switzerland & 43810 & SVP & Swiss Peoples Party \\ 
  Switzerland & 43811 & BDP & Conservative Democratic Party of Switzerland \\ 
  Switzerland & 43951 & FPS & Freedom Party of Switzerland \\ 
  Switzerland & 43953 & Lega & Ticino League \\ 
  UK & 51101 &  & Green \\ 
  UK & 51210 & SF & Sinn F�in \\ 
  UK & 51301 & PC & Plaid Cymru \\ 
  UK & 51320 & Lab & Labour Party \\ 
  UK & 51330 & SDL & Social Democratic Labour Party \\ 
  UK & 51401 & AL & Alliance Party of Northern Ireland \\ 
  UK & 51420 & LP & Liberal Party \\ 
  UK & 51421 & LDP & Liberal Democratic Party \\ 
  UK & 51620 & Con & Conservative Party \\ 
  UK & 51621 & UUP & Ulster Unionist Party \\ 
  UK & 51902 & SNP & Scottish National Party \\ 
  UK & 51903 & DUP & Democratic Union Party \\ 
  UK & 51951 & UKIP & United Kingdom Independence Party \\ 
  Ireland & 53110 & Greens & Ecology party / Green Party \\ 
  Ireland & 53202 &  & Independent \\ 
  Ireland & 53203 & NULL & United Left Alliance \\ 
  Ireland & 53204 & AAA & Anti-Austerity Alliance \\ 
  Ireland & 53205 & AAA-PBP & Anti-Austerity Alliance - People Before Profit \\ 
  Ireland & 53206 & PBP & NULL \\ 
  Ireland & 53220 &  & Workers Party \\ 
  Ireland & 53221 & DL & Democratic Left Party \\ 
  Ireland & 53320 & LP & Labour Party \\ 
  Ireland & 53321 & NULL & Social Democrats \\ 
  Ireland & 53420 & PD & Progressive Democrats \\ 
  Ireland & 53520 & FG & Fine Gael (Family of the Irish) \\ 
  Ireland & 53620 & FF & Fianna Fail (Soldiers of Destiny) \\ 
  Ireland & 53621 & NULL & Renew \\ 
  Ireland & 53951 & Sinn Fein & Sinn Fein (We Ourselves) \\ 
  Malta & 54101 & AD & Democratic Alternative \\ 
  Malta & 54320 & MLP & Malta Labour Party \\ 
  Malta & 54322 & PN & Nationalist Party \\ 
  Czech Republic & 82110 & SZ & Green Party \\ 
  Czech Republic & 82220 & KSCS & Communist Party of Czechoslovakia \\ 
  Czech Republic & 82221 & LB & Left Bloc \\ 
  Czech Republic & 82320 & CSSD & Czechoslovak Party of Social Democracy \\ 
  Czech Republic & 82412 & ODA & Civic Democratic Alliance \\ 
  Czech Republic & 82413 & ODS & Civic Democratic Party-Christian Democratic Party \\ 
  Czech Republic & 82414 & VV & Public Affairs \\ 
  Czech Republic & 82420 & LSU & Liberal Social Union \\ 
  Czech Republic & 82421 & US & Freedom Union \\ 
  Czech Republic & 82520 & CLS & Czechoslovak People�s Party \\ 
  Czech Republic & 82521 & KDS & Christian Democratic Party \\ 
  Czech Republic & 82523 & KDU-CSL & Krest'ansk� a demokratick� unie-Ceskoslovensk� lidov� strana \\ 
  Czech Republic & 82524 & NULL & Coalition \\ 
  Czech Republic & 82601 & SNK & Association of Independents \\ 
  Czech Republic & 82603 & TOP 09 & Tradition, Responsibility, Prosperity 09 \\ 
  Czech Republic & 82710 & SPR-RSC & Rally for the Republic-Republican Party of Czechoslovakia \\ 
  Czech Republic & 82901 & HDZJ & Pensioners' Movement for a Secure Life \\ 
  Czech Republic & 82951 & HSD-SMS & Movement for Self-Governing Democracy-Society for Moravia and Silesia \\ 
  Estonia & 83110 & ER & Estonian Greens \\ 
  Estonia & 83220 & KK & Secure Home \\ 
  Estonia & 83410 & M��d & People's Party Moderates \\ 
  Estonia & 83411 & Kesk & Estonian Centre Party \\ 
  Estonia & 83421 & R & Popular Front \\ 
  Estonia & 83430 & Ref & Estonian Reform Party \\ 
  Estonia & 83610 & P & Right-Wingers \\ 
  Estonia & 83612 & RL & Estonian People's Union \\ 
  Estonia & 83709 & NULL & NULL \\ 
  Estonia & 83710 & Isam & Homeland - Pro Partia Union \\ 
  Estonia & 83711 & ERSP & Estonian National Independence Party  \\ 
  Estonia & 83712 & EK & Estonian Citizens \\ 
  Estonia & 83714 & IRL & Union of Pro Patria and Res Publica \\ 
  Estonia & 83719 & KM� & Coalition Party and Rural Union \\ 
  Estonia & 83810 & EME & Estonian Country People�s Party \\ 
  Estonia & 83901 & SK & Independent Royalists \\ 
  Estonia & 83951 & MKE & Our Home is Estonia \\ 
  Estonia & 83952 & E�RP & Estonian United People's Party \\ 
  Poland & 92021 & LiD & Left and Democrats \\ 
  Poland & 92210 & SLD & Alliance of the Democratic Left \\ 
  Poland & 92211 &  & Democratic Party \\ 
  Poland & 92212 & SLD-UP & NULL \\ 
  Poland & 92320 &  & Labour Solidarity \\ 
  Poland & 92321 &  & Solidarity \\ 
  Poland & 92322 & UP & Labour Union \\ 
  Poland & 92330 & UD &  \\ 
  Poland & 92410 & PUS & Democratic Union \\ 
  Poland & 92420 & KLD & Liberal Democratic Congress \\ 
  Poland & 92431 & PPPP & Polish Friends of Beer Party \\ 
  Poland & 92432 & UPR & Unionof Political Realism \\ 
  Poland & 92433 & KLD &  \\ 
  Poland & 92434 & UW & Freedom Union \\ 
  Poland & 92435 & PO & Civic Platform \\ 
  Poland & 92436 & PiS & Law and Justice \\ 
  Poland & 92440 & RP & Palikot's Movement \\ 
  Poland & 92441 & PJN & Poland Comes First \\ 
  Poland & 92520 & PSChD & Christian Democracy \\ 
  Poland & 92521 & POC & Civic Centre Alliance \\ 
  Poland & 92522 & PChD & Party of Christian Democrats \\ 
  Poland & 92530 & WAK & Catholic Election Action \\ 
  Poland & 92610 & NULL & Congress of the New Right \\ 
  Poland & 92620 & AWS & Solidarity Election Action \\ 
  Poland & 92621 & ROP & Movement for Rebuilding Poland \\ 
  Poland & 92622 & SRP & Self Defense \\ 
  Poland & 92710 & KPN & Confederation for an Independent Poland \\ 
  Poland & 92712 &  & Party X \\ 
  Poland & 92713 & LPR & League of Polish Families \\ 
  Poland & 92810 & PL & Peasant Alliance \\ 
  Poland & 92811 & PSL & Polish Peasant Party \\ 
  Poland & 92901 & BBWR & Non Party Reform Block \\ 
  Poland & 92952 & RAS & Bewegung fur die Autonomie Schlesiens \\ 
  Poland & 92953 &  & German Minority \\ 
  Poland & 92992 &  & Coalition for the Republic \\ 
   \hline
\end{longtable}
\end{footnotesize}


\end{landscape}

\clearpage


\begin{figure}[ht]
	\centering
		\includegraphics[width=1\textwidth]{at_nat.eps}
\end{figure}

\clearpage

\begin{figure}[ht]
	\centering
		\includegraphics[width=1\textwidth]{be_nat.eps}
\end{figure}


\clearpage


\begin{figure}[ht]
	\centering
		\includegraphics[width=1\textwidth]{cz_nat.eps}
\end{figure}


\clearpage


\begin{figure}[ht]
	\centering
		\includegraphics[width=1\textwidth]{dk_nat.eps}
\end{figure}


\clearpage


\begin{figure}[ht]
	\centering
		\includegraphics[width=1\textwidth]{ee_nat.eps}
\end{figure}


\clearpage


\begin{figure}[ht]
	\centering
		\includegraphics[width=1\textwidth]{fi_nat.eps}
\end{figure}

\clearpage


\begin{figure}[ht]
	\centering
		\includegraphics[width=1\textwidth]{fr_nat.eps}
\end{figure}


\clearpage

\begin{figure}[ht]
	\centering
		\includegraphics[width=1\textwidth]{de_nat.eps}
\end{figure}


\clearpage

\begin{figure}[ht]
	\centering
		\includegraphics[width=1\textwidth]{ie_nat.eps}
\end{figure}


\clearpage


\begin{figure}[ht]
	\centering
		\includegraphics[width=1\textwidth]{it_nat.eps}
\end{figure}


\clearpage


\begin{figure}[ht]
	\centering
		\includegraphics[width=1\textwidth]{lu_nat.eps}
\end{figure}


\clearpage


\begin{figure}[ht]
	\centering
		\includegraphics[width=1\textwidth]{mt_nat.eps}
\end{figure}


\clearpage


\begin{figure}[ht]
	\centering
		\includegraphics[width=1\textwidth]{nl_nat.eps}
\end{figure}


\clearpage


\begin{figure}[ht]
	\centering
		\includegraphics[width=1\textwidth]{no_nat.eps}
\end{figure}


\clearpage


\begin{figure}[ht]
	\centering
		\includegraphics[width=1\textwidth]{pl_nat.eps}
\end{figure}


\clearpage


\begin{figure}[ht]
	\centering
		\includegraphics[width=1\textwidth]{pt_nat.eps}
\end{figure}



\clearpage


\begin{figure}[ht]
	\centering
		\includegraphics[width=1\textwidth]{es_nat.eps}
\end{figure}


\clearpage

\begin{figure}[ht]
	\centering
		\includegraphics[width=1\textwidth]{se_nat.eps}
\end{figure}


\clearpage

\begin{figure}[ht]
	\centering
		\includegraphics[width=1\textwidth]{ch_nat.eps}
\end{figure}


\clearpage

\begin{figure}[ht]
	\centering
		\includegraphics[width=1\textwidth]{uk_nat.eps}
\end{figure}



\clearpage


\begin{landscape}

%\subsection{Regional documents}

% latex table generated in R 3.4.1 by xtable 1.8-2 package
% Wed Jun 27 10:12:34 2018
\begin{footnotesize}
\begin{longtable}{lrll}
\caption{Parties issuing regional manifestos covered} \\
Country & CMP code & Acronym & Party \\ 
  \hline
Sweden & 11110 & MP & Green Ecology Party \\ 
  Sweden & 11220 & VP & Left Party \\ 
  Sweden & 11320 & SAP & Social Democratic Labor Party \\ 
  Sweden & 11420 & FP & Liberal People�s Party \\ 
  Sweden & 11520 & KdS & Christian Democratic Community Party \\ 
  Sweden & 11620 & MSP & Moderat Coalition Party \\ 
  Sweden & 11810 & CP & Centre Party \\ 
  Belgium & 21001 & CD\&V/SP.A/VLD & NULL \\ 
  Belgium & 21111 & Ecolo & Francophone Ecologists \\ 
  Belgium & 21112 & Groen & Green! \\ 
  Belgium & 21221 & SPA & Spirit - Socialist Party Different \\ 
  Belgium & 21321 & SP & Flemish Socialist Party \\ 
  Belgium & 21322 & PS & Francophone Socialist Party \\ 
  Belgium & 21401 & Spirit/SLP & Social-liberal Party \\ 
  Belgium & 21403 & LDD & List Dedecker \\ 
  Belgium & 21421 & VLD & Flemish Liberals and Democrats \\ 
  Belgium & 21426 & MR & Reformist Movement \\ 
  Belgium & 21523 & CD\&V & Christian Democratic and Flemish \\ 
  Belgium & 21524 & CDH & Humanist Democratic Centre \\ 
  Belgium & 21914 & VB & Flemish Bloc \\ 
  Belgium & 21916 & N-VA & New-Flemish Alliance \\ 
  Netherlands & 22110 & GL & Green Left \\ 
  Netherlands & 22220 & SP & Socialist Party \\ 
  Netherlands & 22320 & PvdA & Labour Party \\ 
  Netherlands & 22330 & D66 & Democrats '66 \\ 
  Netherlands & 22420 & VVD & People's Party for Freedom and Democracy \\ 
  Netherlands & 22430 & LN & Livable Netherlands \\ 
  Netherlands & 22502 & SGP & Political Reformed Party \\ 
  Netherlands & 22521 & CDA & Christian Democratic Appeal \\ 
  Netherlands & 22526 & CU & ChristianUnion \\ 
  Netherlands & 22601 & PVV & Party for Freedom \\ 
  Netherlands & 22906 & 50+ & 50PLUS \\ 
  Netherlands & 22952 & PvdD & Party for the Animals \\ 
  Netherlands & 22957 & PvhN & Party for the North \\ 
  Netherlands & 22958 & FNP & Frisian National Party \\ 
  Spain & 33103 & EQUO & NULL \\ 
  Spain & 33205 & AGE & Galicial Alternative of the Left \\ 
  Spain & 33210 & Pod & We Can \\ 
  Spain & 33220 & IU & United Left \\ 
  Spain & 33320 & PSOE & Spanish Socialist Workers� Party \\ 
  Spain & 33326 & PSC & Socialists Party of Catalonia \\ 
  Spain & 33402 & Pi & PI - Proposal for the Isles \\ 
  Spain & 33501 & Unio & Democratic Union of Catalonia \\ 
  Spain & 33610 & PP & Popular Party \\ 
  Spain & 33611 & CiU & Convergence and Unity \\ 
  Spain & 33613 & UPN & Union for a Popular Movement \\ 
  Spain & 33902 & PNV/EAJ & Basque Nationalist Party \\ 
  Spain & 33903 & EA & Basque Solidarity \\ 
  Spain & 33904 & PAR & Aragonese Regionalist Party \\ 
  Spain & 33907 & CC & Canarina Coaliton \\ 
  Spain & 33908 & BNG & Galician Nationalist Bloc \\ 
  Spain & 33909 & EHB & Basque Country Unite \\ 
  Spain & 33910 & NC & New Canaries \\ 
  Spain & 33911 & CC & Constituent Call \\ 
  Spain & 33912 & NULL & Together for Yes \\ 
  Spain & 33950 & CA & NULL \\ 
  Spain & 33960 & NULL & Catalonia Yes We Can \\ 
  Spain & 33963 & NULL & NULL \\ 
  Spain & 33965 &  &  \\ 
  Spain & 33967 & Cs & Citizens \\ 
  Spain & 33968 & NULL & NULL \\ 
  Spain & 33969 & NULL & NULL \\ 
  Spain & 33972 & NULL & Commitment \\ 
  Spain & 33992 & UPyD & Union, Progress and Democracy \\ 
  Spain & 33993 &  &  \\ 
  Spain & 33997 & NULL & NULL \\ 
  Germany & 41001 & NULL & Coalition Agreement \\ 
  Germany & 41111 & GR�NE & The Greens \\ 
  Germany & 41112 & GR�NE & Greens/Alliance�90 \\ 
  Germany & 41113 & GR�NE & Alliance�90/Greens \\ 
  Germany & 41221 & PDS & Party of Democratic Socialism \\ 
  Germany & 41222 & L-PDS & The Left.Party of Democratic Socialism \\ 
  Germany & 41223 & LINKE & The Left \\ 
  Germany & 41301 & WASG & Labour and Social Justice � The Electoral Alternative \\ 
  Germany & 41320 & SPD & Social Deomocratic Party of Germany \\ 
  Germany & 41420 & FDP & Free Democratic Party \\ 
  Germany & 41430 & DSU & German Social Union \\ 
  Germany & 41440 & FW & Free Voters \\ 
  Germany & 41441 & NLI & New Liberal \\ 
  Germany & 41521 & CDU & Christian Democratic Union \\ 
  Germany & 41523 & CSU & Christian Social Union of Bavaria \\ 
  Germany & 41601 & ALFA & Alliance for Progress and Renewal \\ 
  Germany & 41610 & NULL & NULL \\ 
  Germany & 41701 & REP & The Republicans \\ 
  Germany & 41702 & NPD & National Democratic Party of Germany \\ 
  Germany & 41703 & DVU & The Germans People Union \\ 
  Germany & 41704 & Schill & Law and Order Offensive Party, Schill-Party \\ 
  Germany & 41705 & BIW & Citizens in Rage \\ 
  Germany & 41901 & AFB & Jobs for Bremen and Bremerhaven \\ 
  Germany & 41902 & SSW & South Schleswig Voter Federation \\ 
  Germany & 41950 & NULL & Pirate Party Germany \\ 
  Germany & 41953 & AfD & Alternative for Germany \\ 
  Austria & 42001 & �VP + Die Gr�nen  &  \\ 
  Austria & 42110 & GA & Green Alternative \\ 
  Austria & 42220 & KP� & Communist Party \\ 
  Austria & 42320 & SP� & Austrian Social Democratic Party \\ 
  Austria & 42420 & FP� & Freedom Movement \\ 
  Austria & 42422 & BZ� & Alliance for the Future of Austria \\ 
  Austria & 42450 & neos & NULL \\ 
  Austria & 42520 & �VP & Austrian People�s Party \\ 
  Austria & 42710 & NULL & NULL \\ 
  Austria & 42952 & FRITZ &  \\ 
  Austria & 42953 & FRANK & NULL \\ 
  Switzerland & 43110 & GPS & Green Party of Switzerland \\ 
  Switzerland & 43120 & GLP & Green Liberal Party of Switzerland \\ 
  Switzerland & 43320 & SPS/PSS & Social Democratic Party \\ 
  Switzerland & 43420 & FDP/PRD & Radical Democratic Party \\ 
  Switzerland & 43520 & CVP/PDC & Christian Democratic People�s Party \\ 
  Switzerland & 43530 & EVP/PEP & Protestant Peoples Party \\ 
  Switzerland & 43532 & AL & Alternative List \\ 
  Switzerland & 43540 & CSP & Christian-social party \\ 
  Switzerland & 43711 & EDU & Federal Democracy Union \\ 
  Switzerland & 43810 & SVP & Swiss Peoples Party \\ 
  Switzerland & 43811 & BDP & Conservative Democratic Party of Switzerland \\ 
  Switzerland & 43954 & Pir & Pirate Party \\ 
  Switzerland & 43994 & MCVD & Vaudian citizen movement \\ 
  UK & 51001 & Coalition (SLP \& SLD) & NULL \\ 
  UK & 51101 &  & Green \\ 
  UK & 51102 &  & Scottish Green Party \\ 
  UK & 51201 & SSP & Scottish Socialist Party \\ 
  UK & 51301 & PC & Plaid Cymru \\ 
  UK & 51320 & Lab & Labour Party \\ 
  UK & 51402 & NULL & Libertarian Party \\ 
  UK & 51421 & LDP & Liberal Democratic Party \\ 
  UK & 51620 & Con & Conservative Party \\ 
  UK & 51902 & SNP & Scottish National Party \\ 
  UK & 51951 & UKIP & United Kingdom Independence Party \\ 
  UK & 51953 & RISE & Respect, Independence, Socialism, Environmentalism \\ 
  Czech Republic & 82110 & SZ & Green Party \\ 
  Czech Republic & 82220 & KSCS & Communist Party of Czechoslovakia \\ 
  Czech Republic & 82320 & CSSD & Czechoslovak Party of Social Democracy \\ 
  Czech Republic & 82402 & ApK & Alternative for the regionns \\ 
  Czech Republic & 82403 & KOA (SOS, VMP, ED)  & Coalition (SOS, VMP, ED) \\ 
  Czech Republic & 82413 & ODS & Civic Democratic Party-Christian Democratic Party \\ 
  Czech Republic & 82523 & KDU-CSL & Krest'ansk� a demokratick� unie-Ceskoslovensk� lidov� strana \\ 
  Czech Republic & 82601 & SNK & Association of Independents \\ 
  Czech Republic & 82603 & TOP 09 & Tradition, Responsibility, Prosperity 09 \\ 
  Czech Republic & 82901 & HDZJ & Pensioners' Movement for a Secure Life \\ 
  Czech Republic & 82902 & S.cz  & NULL \\ 
  Czech Republic & 82903 & SPL  & NULL \\ 
  Czech Republic & 82904 & JIH & NULL \\ 
  Czech Republic & 82905 & HNHRM & NULL \\ 
  Czech Republic & 82906 & V�CHODO?E�I & NULL \\ 
  Czech Republic & 82907 & ZM?NA & NULL \\ 
  Czech Republic & 82908 & NEZ & NULL \\ 
  Czech Republic & 82995 & Koalice & Coalition (KDU-?SL + SNK ED + Nestrann�ci) \\ 
   \hline
\end{longtable}
\end{footnotesize}


\end{landscape}

\clearpage 


\begin{figure}[ht]
	\centering
		\includegraphics[width=1\textwidth]{at_reg.eps}
\end{figure}

\begin{figure}[ht]
	\centering
		\includegraphics[width=1\textwidth]{be_reg.eps}
\end{figure}

\begin{figure}[ht]
	\centering
		\includegraphics[width=1\textwidth]{cz_reg.eps}
\end{figure}

\begin{figure}[ht]
	\centering
		\includegraphics[width=1\textwidth]{de_reg.eps}
\end{figure}

\begin{figure}[ht]
	\centering
		\includegraphics[width=1\textwidth]{es_reg.eps}
\end{figure}

\begin{figure}[ht]
	\centering
		\includegraphics[width=1\textwidth]{nl_reg.eps}
\end{figure}

\begin{figure}[ht]
	\centering
		\includegraphics[width=1\textwidth]{se_reg.eps}
\end{figure}

\begin{figure}[ht]
	\centering
		\includegraphics[width=1\textwidth]{ch_reg.eps}
\end{figure}

\begin{figure}[ht]
	\centering
		\includegraphics[width=1\textwidth]{uk_reg.eps}
\end{figure}


%\begin{thebibliography}{9}

%\end{thebibliography}


\end{document}
